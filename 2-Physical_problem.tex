\section{The physical problem}\label{sec:problem}
Quantum dots are nanoscopic crystals usually from semiconducting materials being of a size to exhibit quantum mechanical properties. In this report we will look at systems of electrons confined in a harmonic oscillator, which acts like a trap. These systems can be considered as quantum dots. In order to study such system, we have to look at the Hamiltonian, which consists of two parts:
\begin{equation}
\hat{H} = \hat{H_0} + \hat{H_1},
\end{equation}
where $\hat{H_0}$ describes the standard harmonic oscillator and $\hat{H_1}$ the repulsive term between the charged particles (in this case electrons). We can than write
\begin{equation}
\hat{H} = \sum_{i=1}^N \left( -\frac{1}{2} \nabla_i^2 + \frac{1}{2} \omega^2 r_i^2 \right) + \sum_{i<j} \frac{1}{r_{ij}}
\end{equation}
The quantity $N$ denotes the number of charged particles, $\omega$ is the oscillator frequency, $r_i$ is the position of particle $i$ given by $r_i = \sqrt{r_{ix}^2 + r_{iy}^2}$ and the distance between two particles is referred to as $r_{ij} = \sqrt{\mathbf{r_1}- \mathbf{r_2}}$.\\
\subsection{Unperturbed wave functions}\label{sec:unperturbed}
First we will take a good at the simpler case of unperturbed particles, not intercating with each other. A case, that is also called a pure harmonic oscillator, because there are no other potential terms in the Hamiltonian, than the oscillator's potential.\\
From quantum mechanics we know, that the wave function corresponding to this Hamiltonian for one electron in two dimensions $(x,y)$ is
\begin{equation}\label{glg:wavefunc1}
\phi_{n_x,n_y}(x,y) = A H_{n_x} (\sqrt{\omega} x) H_{n_y} (\sqrt{\omega} y) \exp\left[-\frac{\omega}{2} (x^2+y^2)\right].
\end{equation}
In this equation the Hermite polynomials $H_{n_x} (\sqrt{\omega} x)$ appear as well as a constant $A$ responsible for the normalization.\\
To get the energy $E$, we have to consider the Eigenvalue equation
\begin{equation}\label{glg:eigenvalue}
\hat{H} \phi_\lambda = E \phi_\lambda,
\end{equation}
which leads to the energy
\begin{equation}
E_{n_x,n_y} = \omega(n_x + n_y +1)
\end{equation}
Looking at the lowest energy state we have $E_{(1)}=\omega$.\\
Advancing now to the case of two electrons who do not repell each other, we have two independent Hamiltonians, one for each electron, and as a result two independent eigenvalue problems as in equation~\ref{glg:eigenvalue}. This leads to the same energy as before, but this time the energy must be taken into account twice. So we get:
\begin{equation}
E_{(2)} = E_{(1)} + E_{(1)} = 2 E_{(1)} = 2\omega.
\end{equation}
The corresponding wave function is then given by
\begin{equation}
\Phi(\mathbf{r_1},\mathbf{r_2}) = C \exp\left[-\frac{\omega}{2} (r_1^2+r_2^2)\right].
\end{equation}
As in equation~\ref{glg:wavefunc1}, $C$ is the normalization constant.\\
Since the particles we are considering are fermions, we must regard the spin as well. The overall spin must be zero, because according to the Pauli principle, two fermions cannot have the same quantum numbers. The electrons we are looking at have exactly the same energy of $E = \omega$, so their only way of obeying the princple is to have different spins. There are two possible states:
\begin{equation}
|\uparrow \downarrow > \text{~and~} |\downarrow \uparrow >
\end{equation}
Hence one of the electrons has spin 1/2 and the other one has spin -1/2. This lead to an overall spin of zero. Despite this they are indistinguishable, which means both states are possible.\\
Regarding the wave function we will now simplificate it by a separational ansatz:
\begin{align}
\psi (\mathbf{r_1,\sigma_1,r_2, \sigma_2}) = \phi_1(\mathbf{r_1}) \phi_2(\mathbf{r_2}), 
\end{align}
which is also known as the Hartree product. Considering fermions, this is not an appropriate way to represent the wave function, because it does not obey the Pauli exclusion principle, which yields to antisymmetric wave functions. To solve this problem, we can use a linear combination of such wave functions $\phi_{1,2}$:
\begin{equation}
\psi (\mathbf{r_1,r_2}) = \frac{1}{\sqrt{2}}\left(\phi_1(\mathbf{r_1}) \phi_2(\mathbf{r_2})-\phi_1(\mathbf{r_2}) \phi_2(\mathbf{r_1})\right), 
\end{equation}
This can also be displayed as a determinant:
\begin{equation}
\psi (\mathbf{r_1,r_2}) = \frac{1}{\sqrt{2}} 
\begin{vmatrix}
\phi_1(\mathbf{r_1}) &\phi_1(\mathbf{r_2})\\
\phi_2(\mathbf{r_1}) & \phi_2(\mathbf{r_2})
\end{vmatrix}
\end{equation}
Since we are going to analyse systems of up to six electrons as well, we will generalize this method introducing the Slater determinant $\mathbf{S}$ for $N$ electrons:
\begin{align}
\psi (\mathbf{r_1,r_2,\cdots,r_N}) = \frac{1}{\sqrt{N}} |\mathbf{S}| = \frac{1}{\sqrt{N}}
\begin{vmatrix}
\phi_1(r_1) & \phi_1(r_2) & \cdots & \phi_1(r_N)\\
\phi_2(r_1) & \ddots & &\vdots \\
\vdots & & \ddots& \vdots\\
\phi_N(r_1) & \cdots & \cdots & \phi_N(r_N)
\end{vmatrix}
\end{align}
For the calculation, it is convenient to reexpress the Slater determinant in terms of $3\times 3$-determinants according to Cramer's rule:
\begin{align}
\begin{vmatrix}
\phi_1(r_1) & \phi_1(r_2) & \cdots & \phi_1(r_6)\\
\phi_2(r_1) & \ddots & &\vdots \\
\vdots & & \ddots& \vdots\\
\phi_6(r_1) & \cdots & \cdots & \phi_6(r_6)
\end{vmatrix}
\propto
\begin{vmatrix}
\phi_1(r_1)& \phi_1(r_2)& \phi_1(r_3)\\
\phi_3(r_1)& \phi_3(r_2)& \phi_3(r_3)\\
\phi_5(r_1)& \phi_5(r_2)& \phi_5(r_3)
\end{vmatrix}
\cdot
\begin{vmatrix}
\phi_2(r_4)& \phi_2(r_5)& \phi_2(r_6)\\
\phi_4(r_4)& \phi_4(r_5)& \phi_4(r_6)\\
\phi_6(r_4)& \phi_6(r_5)& \phi_6(r_6)
\end{vmatrix}
\end{align} 
and to compute their determinants afterwards. What we then get is:
\begin{equation}
\Psi(\mathbf{r_1,r_2,\cdots, r_6})= |\mathbf{S\uparrow}||\mathbf{S\downarrow}|.
\end{equation}
The arrows indicate whether we are looking at spin-up or spin-down particles. We are free to choose which three particles should have spin up and which have spin down. The important thing is, that the overall spin is zero and that two electrons fill up the lowest energy state at $E=1\omega$ (with spin zero) and the other four electrons fill up the second lowest energy state at $E=2\omega$ again with overall spin zero.\\
For six electrons in a pure harmonic oscillator the ground state energy should be at
\begin{equation}
E = 2 \cdot 1\omega + 4 \cdot 2\omega = 10\omega.
\end{equation}
\subsection{Perturbed wave functions}
Furthermore, we are interested at interactions between the electrons: Because of their charge there is a repulsive force between the electrons represented by $\hat{H_1}$ in the Hamiltonian. This must also be considered in the wave function and makes it not analytically solvable. Further explanation on this will be given in~\ref{sec:VMC}.