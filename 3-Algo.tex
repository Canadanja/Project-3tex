\section{The method}\label{sec:algo}
As presented in the previous section the eigenvalue problem of two electrons in a harmonic oscillator without any interaction terms can be easily solved. Looking at electrons in a quantum dot this calculation gains complexity. This is why we introduce the Variational Monte Carlo Method for estimating the electron states.
\subsection{Variational Monte Carlo method}
In the Variational principle, we take the eigenvalue problem form equation~\ref{glg:eigenvalue} and expand the wave function as following:
\begin{equation}
\varphi_0 = \sum_{\lambda=0}^{\infty} c_{0 \lambda} \psi_{\lambda},
\end{equation}
where $c_{0 \lambda}$ are coefficients.\\
In quantum mechanics the energy is the expectation value
\begin{align}
E = &\frac{\langle \psi_0 | \hat{H} | \psi_0 \rangle}{\langle \psi_0 | \psi_0 \rangle}\\
\intertext{So when we apple the expansion to this, we get:}
&\frac{\langle \varphi_0 | \hat{H} | \varphi_0 \rangle}{\langle \varphi_0 | \varphi_0 \rangle}\\
= &\frac{\sum_{\alpha,\beta}c_{0\alpha}^*c_{0 \beta} \int d\tau \psi_{\alpha}^*(\tau)\hat{H}\psi_{\beta}(\tau)}{\sum_{\alpha,\beta}c_{0\alpha}^*c_{0 \beta} \int d\tau \psi_{\alpha}^*(\tau)\psi_{\beta}(\tau)}\\
= &\frac{\sum_{\alpha} E_{\alpha} |c_{0\alpha}|^2}{\sum_{\alpha} |c_{0\alpha}|^2},
\end{align}
because by construction $\langle\psi_{\alpha}| \psi_{\beta}\rangle = \delta_{\alpha \beta}$ for eigenfunctions $\psi_{\alpha},\psi_{beta}$.\\
We have to consider two cases now:
\begin{itemize}
\item If the expansion $\varphi_0$ is not the eigenfunction $\psi_0$ we get the an energy
\begin{equation}
E_0 \leqslant \frac{\langle \varphi_0 | \hat{H} | \varphi_0 \rangle}{\langle \varphi_0 | \varphi_0 \rangle}.
\end{equation}
\item If the expansion $\varphi_0$ is exact the eigenfunction $\psi_0$ we get the exact energy
\begin{equation}
E_0 = \frac{\langle \varphi_0 | \hat{H} | \varphi_0 \rangle}{\langle \varphi_0 | \varphi_0 \rangle}.
\end{equation}
\end{itemize}
In the second case the variance of the energy
\begin{equation}
\mathrm{var}(E) = \langle H^2 \rangle - \langle H\rangle^2 = 0.
\end{equation}
As the expansion wave function $\varphi_0$ we use a trial wave function we will call $\psi_T (\mathbf{r_1, r_2},\alpha, /beta)$, with $\alpha$ and $\beta$ being the variational parameters. In this report the trial wave function for two electrons has the form:
\begin{equation}
\psi_T(\mathbf{r_1,r_2}) = C \exp\left[-\frac{\omega}{2} (r_1^2+r_2^2)\right] \exp \left[ \frac{a_{12} r_{12}}{(1+\beta r_{12})} \right]
\end{equation}
with
\begin{align}
a_{12} =\left\{\begin{array}{cl} 1, & \mbox{for} \uparrow\downarrow\\ 1/3, & \mbox{for} \uparrow\uparrow,\downarrow\downarrow \end{array}\right.
\end{align}
And
\begin{equation}
r_{12} = \sqrt{\mathbf{r_1} - \mathbf{r_2}}
\end{equation}
 Then we will compute the expectation value $E(\alpha)$ and find its minimum or alternatively the minimum of its variance $\mathrm{var}(E(\alpha))$...
\subsection{Monte Carlo methods}
\subsection{Metropolis algorithm}